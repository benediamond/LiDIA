%%%%%%%%%%%%%%%%%%%%%%%%%%%%%%%%%%%%%%%%%%%%%%%%%%%%%%%%%%%%%%%%%
%%
%%  gf_element       LiDIA documentation
%%
%%  This file contains the documentation of the class gf_element
%%
%%  Copyright   (c)   1995-1999   by  LiDIA-Group
%%
%%  Author:  Detlef Anton, Thomas Pfahler
%%

\newcommand{\ff}{\mathit{ff}}


%%%%%%%%%%%%%%%%%%%%%%%%%%%%%%%%%%%%%%%%%%%%%%%%%%%%%%%%%%%%%%%%%

\NAME

\CLASS{gf_element} \dotfill an element over a finite field


%%%%%%%%%%%%%%%%%%%%%%%%%%%%%%%%%%%%%%%%%%%%%%%%%%%%%%%%%%%%%%%%%

\ABSTRACT

\code{gf_element} is a class that provides arithmetics over a finite field.  If the finite field
is defined by a polynomial $f(X) \bmod p$, then a variable of type \code{gf_element} holds a
representation of a field element by its polynomial representation.


%%%%%%%%%%%%%%%%%%%%%%%%%%%%%%%%%%%%%%%%%%%%%%%%%%%%%%%%%%%%%%%%%

\DESCRIPTION

A variable of type \code{gf_element} consists of an instance of type \code{galois_field}
(indicating the field over which the element is defined) and the representation of the element.
The internal representation varies for different classes of finite fields.  Currently, we
distinguish finite prime fields, extension fields of odd characteristic, and extension fields of
characteristic 2.  However, the interface of the class \code{gf_element} is completely
independent of the internal representation.

Let $e$ be an instance of type \code{gf_element}.  If $\ff = \code{$e$.get_field()}$ and $g$
denotes the polynomial \code{$\ff$.irred_polynomial()}, then $e$ represents the element
($\code{$e$.polynomial_rep()} \bmod g(X)) \in \bbfF_p[X]/(g(X)\bbfF_p[X])$, where $p =
\code{$\ff$.characteristic()}$.


%%%%%%%%%%%%%%%%%%%%%%%%%%%%%%%%%%%%%%%%%%%%%%%%%%%%%%%%%%%%%%%%%

\CONS

\begin{fcode}{ct}{gf_element}{}
  initializes the \code{gf_element} over a mainly useless dummy field.
\end{fcode}

\begin{fcode}{ct}{gf_element}{const galois_field & $K$}
  constructs an element over the finite field $K$.  The element is initialized with zero.
\end{fcode}

\begin{fcode}{ct}{gf_element}{const gf_element & $e$}
  constructs a copy of $e$.
\end{fcode}

\begin{fcode}{dt}{~gf_element}{}
\end{fcode}


%%%%%%%%%%%%%%%%%%%%%%%%%%%%%%%%%%%%%%%%%%%%%%%%%%%%%%%%%%%%%%%%%

\ACCS

Let \code{e} be an instance of type \code{gf_element}.

\begin{cfcode}{galois_field}{$e$.get_field}{}
  returns the field over which the element \code{e} is defined.
\end{cfcode}

\begin{cfcode}{const Fp_polynomial &}{$e$.polynomial_rep}{}
  returns the polynomial representation of the element \code{e}.
\end{cfcode}


%%%%%%%%%%%%%%%%%%%%%%%%%%%%%%%%%%%%%%%%%%%%%%%%%%%%%%%%%%%%%%%%%

\ASGN

Let \code{e} be an instance of type \code{gf_element}.

\begin{fcode}{void}{$e$.set_polynomial_rep}{const Fp_polynomial & $g$}
  sets $e$ to the element $g \bmod \code{$e$.get_field().irred_polynomial()}$.
\end{fcode}

\begin{fcode}{void}{$e$.assign_zero}{}
  $e \assign 0$.
\end{fcode}

\begin{fcode}{void}{$e$.assign_one}{}
  $e \assign 1$.
\end{fcode}

\begin{fcode}{void}{$e$.assign_zero}{const galois_field & $K$}
  $e$ is assigned the zero element of the field $K$.
\end{fcode}

\begin{fcode}{void}{$e$.assign_one}{const galois_field & $K$}
  $e$ is assigned the element 1 of the field $K$.
\end{fcode}

\begin{fcode}{void}{$e$.assign}{const bigint & $a$}
  $e$ is assigned the element $a \cdot 1$ of the field $\ff$.
\end{fcode}

The operator \code{=} is overloaded.  For efficiency reasons, the following function is also
implemented:

\begin{fcode}{void}{$e$.assign}{const gf_element & $a$}
  $e \assign a$.
\end{fcode}


%%%%%%%%%%%%%%%%%%%%%%%%%%%%%%%%%%%%%%%%%%%%%%%%%%%%%%%%%%%%%%%%%

\ARTH

Let \code{e} be an instance of type \code{gf_element}.

The following operators are overloaded and can be used in exactly the same way as for machine
types in C++ (e.g.~\code{int}) :

\begin{center}
  \code{(binary) +, -, *, /, +=, -=, *=, /=}\\
\end{center}

To avoid copying, these operations can also be performed by the following functions:

\begin{fcode}{void}{add}{gf_element & $c$, const gf_element & $a$, const gf_element & $b$}
  $c \assign a + b$.
\end{fcode}

\begin{fcode}{void}{add}{gf_element & $c$, const bigint & $a$, const gf_element & $b$}
  $c \assign a \cdot 1 + b$ over the field \code{$b$.get_field()}.
\end{fcode}

\begin{fcode}{void}{add}{gf_element & $c$, const gf_element & $a$, const bigint & $b$}
  $c \assign a + b \cdot 1$ over the field \code{$a$.get_field()}.
\end{fcode}

\begin{fcode}{void}{subtract}{gf_element & $c$, const gf_element & $a$, const gf_element & $b$}
  $c \assign a - b$.
\end{fcode}

\begin{fcode}{void}{subtract}{gf_element & $c$, const bigint & $a$, const gf_element & $b$}
  $c \assign a \cdot 1 - b$ over the field \code{$b$.get_field()}.
\end{fcode}

\begin{fcode}{void}{subtract}{gf_element & $c$, const gf_element & $a$, const bigint & $b$}
  $c \assign a - b \cdot 1$ over the field \code{$a$.get_field()}.
\end{fcode}

\begin{fcode}{void}{multiply}{gf_element & $c$, const gf_element & $a$, const gf_element & $b$}
  $c \assign a \cdot b$.
\end{fcode}

\begin{fcode}{void}{multiply}{gf_element & $c$, const bigint & $a$, const gf_element & $b$}
  $c \assign a \cdot 1 \cdot b$ over the field \code{$b$.get_field()}.
\end{fcode}

\begin{fcode}{void}{multiply}{gf_element & $c$, const gf_element & $a$, const bigint & $b$}
  $c \assign a \cdot b \cdot 1$ over the field \code{$a$.get_field()}.
\end{fcode}

\begin{fcode}{void}{divide}{gf_element & $c$, const gf_element & $a$, const gf_element & $b$}
  $c \assign a / b$, if $b \neq 0$.  Otherwise the \LEH will be invoked.
\end{fcode}

\begin{fcode}{void}{divide}{gf_element & $c$, const bigint & $a$, const gf_element & $b$}
  $c \assign (a \cdot 1) / b$ over the field \code{$b$.get_field()}, if $b \neq 0$.  Otherwise the \LEH
  will be invoked.
\end{fcode}

\begin{fcode}{void}{divide}{gf_element & $c$, const gf_element & $a$, const bigint & $b$}
  $c \assign a / (b \cdot 1)$ over the field \code{$a$.get_field()}, if $b \neq 0$.  Otherwise the \LEH
  will be invoked.
\end{fcode}

\begin{fcode}{void}{$e$.negate}{}
  $e \assign -e$.
\end{fcode}

\begin{fcode}{void}{negate}{gf_element & $a$, const gf_element & $b$}
  $a \assign -b$.
\end{fcode}

\begin{fcode}{void}{$e$.multiply_by_2}{}
  $e \assign 2 \cdot e$.
\end{fcode}

\begin{fcode}{void}{$e$.invert}{}
  $e \assign e^{-1}$, if $e \neq 0$.
Otherwise the \LEH will be invoked.
\end{fcode}

\begin{fcode}{void}{invert}{gf_element & $a$, const gf_element & $b$}
  $a \assign b^{-1}$, if $b \neq 0$.  Otherwise the \LEH will be invoked.
\end{fcode}

\begin{fcode}{gf_element}{inverse}{const gf_element & $a$}
  returns $a^{-1}$, if $a \neq 0$.  Otherwise the \LEH will be invoked.
\end{fcode}

\begin{fcode}{gf_element}{sqrt}{const gf_element & $a$}
  returns $x$ with $x^2 = a$.
\end{fcode}

\begin{fcode}{void}{square}{gf_element & $a$, const gf_element & $b$}
  $a \assign b^2$.
\end{fcode}

\begin{fcode}{void}{power}{gf_element & $c$, const gf_element & $a$, const bigint & $e$}
  $c \assign a^e$.
\end{fcode}

\begin{fcode}{void}{pth_power}{gf_element & $c$, const gf_element & $a$, lidia_size_t $e$}
  $c \assign a^{p^e}$, where $p$ is the characteristic of the corresponding field.
\end{fcode}


%%%%%%%%%%%%%%%%%%%%%%%%%%%%%%%%%%%%%%%%%%%%%%%%%%%%%%%%%%%%%%%%%

\COMP

Let \code{e} be an instance of type \code{gf_element}.

The  binary operators \code{==},  \code{!=} are overloaded.

\begin{cfcode}{bool}{$e$.is_zero}{}
  returns \TRUE if $e = 0$, \FALSE otherwise.
\end{cfcode}

\begin{cfcode}{bool}{$e$.is_one}{}
  returns \TRUE if $e = 1$, \FALSE otherwise.
\end{cfcode}


%%%%%%%%%%%%%%%%%%%%%%%%%%%%%%%%%%%%%%%%%%%%%%%%%%%%%%%%%%%%%%%%%

\BASIC

Let \code{e} be an instance of type \code{gf_element}.

\begin{fcode}{void}{$e$.randomize}{}
  sets $e$ to a random element of the field \code{$e$.get_field()}.
\end{fcode}

\begin{fcode}{void}{$e$.randomize}{lidia_size_t $d$}
  sets $e$ to a random element of the subfield $K$ of \code{$e$.get_field()}, where $K$ has
  (absolute) degree $d$.  If $d$ is not a divisor of \code{$e$.get_field().degree()},
  then the \LEH is invoked.
\end{fcode}

\begin{fcode}{void}{swap}{gf_element & $a$, gf_element & $b$}
  swaps the values of $a$ and $b$.
\end{fcode}

\begin{fcode}{udigit}{hash}{const gf_element & $a$}
  returns a hash value for $a$.
\end{fcode}


%%%%%%%%%%%%%%%%%%%%%%%%%%%%%%%%%%%%%%%%%%%%%%%%%%%%%%%%%%%%%%%%%

\HIGH

Let $e$ be an instance of type \code{gf_element}.

\begin{cfcode}{bigint}{$e$.order}{}
  returns the order of $e$ in the multiplicative group of \code{$e$.get_field()}.
\end{cfcode}
% this might take some time because we might need to compute a factorization
% of the order of the multiplicative group...

\begin{cfcode}{multi_bigmod}{$e$.trace}{}
  returns the trace of $e$.
\end{cfcode}

\begin{cfcode}{multi_bigmod}{$e$.norm}{}
  returns the norm of $e$.
\end{cfcode}

\begin{cfcode}{bool}{$e$.is_square}{}
  returns \TRUE if $e$ is a square, \FALSE otherwise.
\end{cfcode}

\begin{cfcode}{bool}{$e$.is_primitive_element}{}
  checks whether the order of $e$ in the multiplicative group is $p^n-1$, where $p$ is the
  characteristic and $n$ is the degree of the corresponding field.
\end{cfcode}

\begin{cfcode}{bool}{$e$.is_free_element}{}
  checks whether $e$ is a free element, i.e. the generator of a normal base.
\end{cfcode}

\begin{fcode}{gf_element}{$e$.assign_primitive_element}{const galois_field & $\ff$}
  sets $e$ to a generator of the multiplicative group of the field $\ff$.

  \code{$e$.assign_primitive_element($ff$)} recomputes a generator on each
  call and therefore is likely to return different values on subsequent
  calls. If you don't need different values on each call, then
  \code{galois_field::generator()} is more efficient.
\end{fcode}

\begin{cfcode}{unsigned int}{$e$.get_absolute_degree}{}
  returns the extension degree $n$ of the current field over which $e$ is defined.
\end{cfcode}

\begin{cfcode}{unsigned int}{$e$.get_relative_degree}{}
  returns the minimal extension degree $k$ of the field $\GF(p^k)$ over the prime field $\GF(p)$
  such that $e$ is an element of $\GF(p^k)$ ($p$ denotes the characteristic of the corresponding
  field).
\end{cfcode}

\begin{cfcode}{bigint}{$e$.lift_to_Z}{}
  If $e$ is an element of a prime field $\GF(p)$ of characteristic $p$, then return the
  least non negative residue of the residue class of the class $e \bmod p$.  Otherwise the \LEH is
  invoked.
\end{cfcode}

\begin{fcode}{bool}{$e$.solve_quadratic}{const gf_element& $a_1$, const gf_element& $a_0$}
  If the polynomial $X^2 + a_1 X + a_0$ has a root over the field over which $a_1$ and $a_0$ are
  defined, then $e$ is set to a root and \TRUE is returned.  If the polynomial does not have a
  root, \FALSE is returned.  If $a_1$ and $a_0$ are defined over different fields, the \LEH is
  invoked.
\end{fcode}


%%%%%%%%%%%%%%%%%%%%%%%%%%%%%%%%%%%%%%%%%%%%%%%%%%%%%%%%%%%%%%%%%

\IO

We distinguish between \emph{verbose} and \emph{short} I/O format.
Let $e$ be an instance of type \code{gf_element}.  The verbose format
is ``(g(X), f(X))'' where $g(X) = \code{$e$.polynomial_rep()}$ and
$f(X) = \code{get_field().irred_polynomial()}$.  In case of a prime
field $\GF(p)\cong\bbfZ/p\bbfZ$, the element $m \bmod p$ can also be
written as \code{(m, p)}.

For the short format, we assume that the finite field over which the
element is defined is already known, i.e. if \code{$e$.get_field()}
does not return the dummy field.  We distinguish the following cases,
depending on the finite field, which we will denote by $K$:
\begin{itemize}
\item $K = \GF(p)$ is a prime field: ``\code{(m)}'' means the element 
$m\bmod p\in K$;
\item $K = \GF(p^n), p \neq 2, n>1$: ``\code{(g(X))}'' means the 
element $g(X) \bmod \code{$K$.irred_polynomial()}$;
\item $K = \GF(2^n)$, $n > 1$: ``\code{(Dec:$d$)}'', and 
``\code{(Hex:$d$)}'' mean the element $\sum_{i} a_i X^i \bmod 2$ for
  $d = \sum_{i} a_i2^i$, where $d$ is an integer in decimal, or
  hexadecimal representation, respectively.
\item $K=GF(p^n)$: ``\code{$d$}'' means the  
element $(\sum_{i} a_i X^i) \bmod \code{$K$.irred_polynomial()}$ for
  $d = \sum_{i} a_ip^i$, where $d$ is an integer in decimal.
\end{itemize}

%\begin{center}
%\begin{tabular}{lll}
%\code{(m)}    &$m\bmod p\in K$ &if $K=\GF(p)$ is a prime field\\
%\code{(g(X))} &$g(X)\bmod {}$\code{K.irred_polynomial()}
%                                       &if $K=\GF(p^n), p\neq2, n>1$\\
%\code{(Dec:d)}        &$\sum\limits_{i} a_iX^i\bmod 2$ for
%                                       $d=\sum\limits_{i}a_i2^i$
%                                       &if $K=\GF(2^n), n>1$\\
%\code{(Hex:d)}        &\multicolumn{2}{l}{analogously, only that \code{d} is expected in hexadecimal representation}
%\end{tabular}
%\end{center}

The \code{istream} operator \code{>>} and the \code{ostream} operator \code{<<} are overloaded.

\begin{fcode}{static void}{set_output_format}{unsigned int $m$}
  sets the output format for all elements of the class \code{gf_element} to \emph{verbose} if
  $m \neq 0$.  Otherwise the output format will be \emph{short}.
\end{fcode}

\begin{fcode}{unsigned int}{get_output_format}{}
  returns 0, if the output format is set to ``short'', and 1 otherwise.
\end{fcode}

\begin{fcode}{istream &}{operator >>}{istream & in, gf_element & $e$}
  reads an element of a finite field from \code{istream} \code{in}.  The input must be of the
  format described above.  If the input is in the short format, the field over which the element
  is defined must already be set beforehand, otherwise the \LEH is invoked.
\end{fcode}

\begin{fcode}{ostream &}{operator <<}{ostream & out, const gf_element & $e$}
  writes the element $e$ to \code{ostream} \code{out} in verbose or short format, according
  to \linebreak\code{gf_element::get_output_format()}.
\end{fcode}


%%%%%%%%%%%%%%%%%%%%%%%%%%%%%%%%%%%%%%%%%%%%%%%%%%%%%%%%%%%%%%%%%

\SEEALSO

\SEE{galois_field}


%%%%%%%%%%%%%%%%%%%%%%%%%%%%%%%%%%%%%%%%%%%%%%%%%%%%%%%%%%%%%%%%%

\EXAMPLES

\begin{quote}
\begin{verbatim}
#include <LiDIA/gf_element.h>

int main()
{
    bigint p;
    lidia_size_t d;

    cout << "Please enter the characteristic ";
    cout << "of the finite field : "; cin >> p;
    cout << "Please enter the degree of the ";
    cout << "finite field : "; cin >> d;

    galois_field field(p, d);
    cout << "This field has ";
    cout << field.number_of_elements();
    cout <<" elements.\n";

    cout << "The defining polynomial of the field is\n";
    field.irred_polynomial().pretty_print();
    cout << endl;

    gf_element elem;
    elem.assign_primitive_element(field);
    cout << elem << " is a primitive element in this field.\n";

    return 0;
}
\end{verbatim}
\end{quote}


%%%%%%%%%%%%%%%%%%%%%%%%%%%%%%%%%%%%%%%%%%%%%%%%%%%%%%%%%%%%%%%%%

\AUTHOR

Detlef Anton, Franz-Dieter Berger, Stefan Neis, Thomas Pfahler
